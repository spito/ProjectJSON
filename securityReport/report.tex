\documentclass[11pt]{article}

% ===== LOADING PACKAGES =====
% language settings, main document language last
\usepackage[english]{babel}
% enabling new fonts support (nicer)
\usepackage{lmodern}
% setting input encoding
\usepackage[utf8]{inputenc}
% setting output encoding
\usepackage[T1]{fontenc}
% set page margins
\usepackage[top=2.5cm, bottom=2.5cm, left=2cm, right=2cm]{geometry}	% pc version
% package to make bullet list nicer
\usepackage{enumitem}
% setting custom colors for links
\usepackage{xcolor}
\definecolor{dark-red}{rgb}{0.6,0.15,0.15}
\definecolor{dark-green}{rgb}{0.15,0.4,0.15}
\definecolor{medium-blue}{rgb}{0,0,0.5}
% generating hyperlinks in document
\usepackage{url}
\usepackage[pdfpagelabels,    % write arabic labels to all pages
            unicode,          % allow unicode characters in links
            colorlinks=true,  % use colored links instead of boxed
            linkcolor={dark-red},
            citecolor={dark-green},
            urlcolor={medium-blue}
            ]{hyperref}

\begin{document}
\title{Security analysis of JSON parser}
\date{\today}
\author{Ravibabu Matta, Martin Ukrop, Jiří Weiser}
\maketitle

\section{Introduction}

We analyzed a JSON parser written in C++ by Vít Šesták and Daniil Leksin. According to the specification, the parser is slightly more permissive than it should, some examples are provided. Nevertheless, the authors claim that no inputs should lead to memory corruption. The specification assumes a valid UTF-8 stream on the input.

\section{Static and dynamic analysis}

Microsoft \textit{PreFast} reports no issues and \textit{CppCheck} has found just a single (but meaningful) style issue: exceptions should be caught by reference, not by value. \textit{Valgrind}'s memcheck tool reports no memory issues on multiple JSON inputs tested.

\section{Manual code inspection}

Manual testing and code ispection revealed the following issues:
\begin{itemize}[itemsep=0pt]
\item The parser is able to parse numbers which are not allowed by JSON standard, such as \texttt{"-.1"}. However, such inputs cause no memory corruption or parsing problems.
\item The parser should check the end-of-file character after the last token. Omiting this check allows for multiple subsequent JSON strings to be parsed. However, such inputs cause no memory corruption or parsing problems.
\item The parser is not able to handle unfinished strings correctly. For instance, the insput string \texttt{"} (single double-quote character) causes the parser to crash with segmentation fault.
\item The parser accepts some string characters inside the range 0x00--0x1F which should be refused. These characters should be encoded in Unicode escape sequence.
\item The parser is not capable of accepting valid Unicode escape sequences of characters larger than 0xFFFF (e.g.\ \texttt{"\textbackslash{}uD852\textbackslash{}uDF62"}). However, such inputs cause no memory corruption or parsing problems.
\item The parser is not capable of accepting valid Unicode escape sequence of some characters less than 0xFFFF. For example, using the Euro sign on the input (\texttt{"\textbackslash{}u20ac"}) causes a problem in the code. The stack is not exposed, only UTF-8 characters higher than 127 are badly flushed.
\item The \texttt{Json::Value} object does not define a virtual destructor. Even though this does not lead to a memory leak in this case, that is only due to a (somewhat improper) usage of \texttt{std::shared\_ptr}.
\end{itemize}

\section{Conclusions}

The automated analysis did not find any major flaws. Manual inspection revealed several issues, bringing into attention the following three major problems:
\begin{itemize}[itemsep=0pt]
\item The function \texttt{shared\_ptr<Json::String> Json::String::readStringFrom(istream \&in)} does not return any value if the ending quotation mark is not reached. This results into uninitialized \texttt{std::shared\_ptr< Json::String >} and therefore a segmentation fault.
\item The function \texttt{void Json::String::dumpTo(ostream \&out,int indent) const} flushes almost all Unicode escape sequences less than 0xFFFF incorrectly.
\item The \texttt{Json::Value} object should define a virtual destructor.
\end{itemize}

\end{document}
