\documentclass[11pt]{article}

% ===== LOADING PACKAGES =====
% language settings, main documnet language last
\usepackage[english]{babel}
% enabling new fonts support (nicer)
\usepackage{lmodern}
% setting input encoding
\usepackage[utf8]{inputenc}
% setting output encoding
\usepackage[T1]{fontenc}
% set page margins
\usepackage[top=2.5cm, bottom=2.5cm, left=2cm, right=2cm]{geometry}	% pc version
% package to make bullet list nicer
\usepackage{enumitem}
% setting custom colors for links
\usepackage{xcolor}
\definecolor{dark-red}{rgb}{0.6,0.15,0.15}
\definecolor{dark-green}{rgb}{0.15,0.4,0.15}
\definecolor{medium-blue}{rgb}{0,0,0.5}
% generating hyperlinks in document
\usepackage{url}
\usepackage[pdfpagelabels,    % write arabic labels to all pages
            unicode,          % allow unicode characters in links
            colorlinks=true,  % use colored links instead of boxed
            linkcolor={dark-red},
            citecolor={dark-green},
            urlcolor={medium-blue}
            ]{hyperref}

\begin{document}
\title{Analysis of FileZilla codebase}
\date{\today}
\author{Ravibabu Matta, Martin Ukrop, Jiří Weiser}
\maketitle

\section{Introduction}

Short intro.

\section{Static analysis using CPPcheck}

Contents to be added.

\section{Static analysis using PreFast}

Contents to be added.

\section{Dynamic analysis using Valgrind}

Running Filezilla with Valgrind \textit{memcheck} tool resulted in several leaks in the program. We inspected a GNOME version of the software, thus we emplyed some common suppression patters to try to exclude leaks caused by external libraries. The suppression files used can be obtained from \url{https://github.com/dtrebbien/GNOME.supp}.

The inspected run consisted of just opening the program and correctly closing it again (\textit{File > Exit}). This resulted in:
\begin{itemize}[topsep=0pt, itemsep=0pt]
\item 6.5 kiB of definitely lost bytes,
\item 33.3 kiB of indirectly lost bytes,
\item 51.9 kiB of possibly lost bytes,
\item 2.4 MiB of bytes still reachable at exit and
\item 13.4 kiB of suppressed bytes.
\end{itemize}
\noindent
The blocks still reachable at exit are of little interest to us, since the optimization strategy of GTK is to leave some object allocated for cleanup during program termination. However, those 91.7 kiB of (possibly or indirectly) lost data should not be there. It may be (at least partly) due to leaks in supporting libraries (mainly GTK). Furthermore, this is only a result for plain opening and closing of the program -- with more interaction the number of leaks may rise.

\section{Code inspection with regards to password processing}

Contents to be added.

\section{Conclusions}

Contents to be added.

\end{document}
